% !TEX root = ../Thesis.tex

\chapter{State of the Art}

The State of the Art chapter is where you discuss \ldots well, the state of the art. Obvious, isn't it?

You have to explain the main research papers related to your thesis and their shortcomings.

It's up to the writer to cite everything or just the sources that are directly relevant to the research topic, and the choice also depends on the kind of thesis work.

For example, one might refer to~\cite{knuth97} as one of the first books about algorithms and programming issues, or~\cite{nash51} for the foundations of Game Theory.

Nevertheless, referring to standards (see~\cite{rfc1925}) for a more accurate technology description is also important.

It is customary to end this chapter by recapping that, given the current state of the art, there is a gap in the knowledge that must be filled - and this is the goal of the present work.

This chapter and the bibliography are an essential part of the thesis. They show that you did your research starting from solid foundations, and they allow the reader to both replicate your results and continue your work.
Hence, the bibliography must be good (relevant works of solid reputation) and correct (allow the reader to find the referenced paper).

One of the best ways to do so is to create your bibliography using a tool, e.g., JabRef\footnote{\url{https://www.jabref.org}}, which will help you in organising the bibliography. JabRef (or an equivalent tool) will help you in creating a bibliographic database (a so-called \texttt{.bib} file). Only the entries effectively cited will be imported into the thesis with the correct citation style.

A more straightforward way to organise the bibliographic entries of the papers, books, or whatever you cite in the thesis is to just write them into a text file named *.bib.

Note that most websites allow you to download the bibliographic entry of a paper (or book, or whatever) directly. For example, IEEEXplore has a button ``Cite This'' that allows you to download the entry and copy-paste it into your bibliographic tool. Just select `BibTeX', copy-paste, and you will have the correct entry (well, mostly, always double-check).