% !TEX root = ../Thesis.tex

\chapter{Tabella di confronto tramite PSNR dei vari modelli  }


\begin{table}[h] % ambiente table per inserire una didascalia
    \centering
    \begin{tabular}{|c|c|c|c|c|}
    \hline
    Bioma & SAR-CAM & FANS & SARBM3D & TESI \\ \hline
    Agricultural &  24.95  & 24.58  & 25.37 & 19.07  \\ \hline
    Airplane & 26.07 & 23.91 & 23.20 & 23.64 \\ \hline
    Baseball diamond & 28.50 & 27.20 & 26.87 &  21.05\\ \hline
    Beach & 28.89 & 25.84 & 24.50 & 16.11 \\ \hline
    Buldings & 24.51 & 22.23 & 21.52 & 21.89 \\ \hline
    Chaparral & 22.93 & 21.49 & 22.43 & 18.78 \\ \hline
    Forest & 26.48 & 25.66 & 25.97 & 18.44 \\ \hline
    Freeway & 25.88 & 24.03 &  23.89 & 21.22 \\ \hline
    Golf course& 28.41 & 27.23 & 26.97 & 20.86 \\ \hline
    Harbor & 23.17 & 21.09 & 20.56 & \\ \hline
    Intersection & 24.98 & 23.51 & 23.44 & \\ \hline
    Mobile homepark& 23.00 & 21.32 & 20.74 & \\ \hline
    Overpass & 25.46 & 23.79 & 23.56 & \\ \hline
    Parkinglot & 22.36 & 21.01 & 20.69 & \\ \hline
    River & 25.76 & 24.86 & 25.03 & \\ \hline
    Runway & 27.22 & 24.88 & 24.82 & \\ \hline
    Sparse residential & 25.27 & 24.01 & 24.00 & \\ \hline
    Storage tanks & 25.67 & 23.63 & 22.91 & \\ \hline
    Tennis court & 25.99 & 24.70 & 24.62 & \\ \hline
    \end{tabular}
    \caption{I valori sopra, indicano la media del PSNR di 100 immagini.
    Ogni modello usato in TESI è stato allenato per 3 epoche con un dataset da 30'000 immagini.
    Mi sono fermato a 3 epoche perchè il valore di loss raggiunge valori molto bassi, tipo 0.005}
    \label{tab:esempio}
    \end{table}
    