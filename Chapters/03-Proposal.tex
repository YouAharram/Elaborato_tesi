% !TEX root = ../Thesis.tex

\chapter{How to heat the water differently}
\label{cap:proposal}

This is the `main' chapter of your thesis.

Here you have to show that \emph{your} version of heated water differs slightly from any other known version of heated water, and this is important.

\section{How did I discover a novel type of heated water}

Explain in detail what are the steps to heat the water in a novel way.

\section{How my heated water differs from the previous ones}

Describe why and how your findings are different from the past versions.

Here you might want to add code (see for example Listing~\ref{code:example}), or tables (see Table~\ref{tab:example}).

Note that figures, listings, tables, and so on, should never be placed `manually'. Let LaTeX decide where to put them - you'll avoid headaches (and bad layouts). Furthermore, each of them must be referred to at least once in the body of the thesis.


\begin{lstlisting}[language=Python, caption=Python example, float, label=code:example]
import numpy as np
    
def incmatrix(genl1,genl2):
    m = len(genl1)
    n = len(genl2)
    M = None #to become the incidence matrix
    VT = np.zeros((n*m,1), int)  #dummy variable
    
    #compute the bitwise xor matrix
    M1 = bitxormatrix(genl1)
    M2 = np.triu(bitxormatrix(genl2),1) 

    for i in range(m-1):
        for j in range(i+1, m):
            [r,c] = np.where(M2 == M1[i,j])
            for k in range(len(r)):
                VT[(i)*n + r[k]] = 1;
                VT[(i)*n + c[k]] = 1;
                VT[(j)*n + r[k]] = 1;
                VT[(j)*n + c[k]] = 1;
                
                if M is None:
                    M = np.copy(VT)
                else:
                    M = np.concatenate((M, VT), 1)
                
                VT = np.zeros((n*m,1), int)
    
    return M
\end{lstlisting}

\begin{table}[tbp]
\centering
\caption{Example table}
\label{tab:example}
\begin{tabular}{@{}lll@{}}
\toprule
Country       & Country code & ISO codes \\ \midrule
Canada        & 1            & CA / CAN  \\
Italy         & 39           & IT / ITA  \\
Spain         & 34           & ES / ESP  \\
United States & 1            & US / USA  \\ \bottomrule
\end{tabular}
\end{table}

