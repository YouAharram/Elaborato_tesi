% !TEX root = ../Thesis.tex

\chapter{Introduzione}

I satelliti SAR sono satelliti dotati di un radar ad apertura sintetica che permette loro di
acquisire immagini della superficie terrestre indipendentemente dalle condizioni meteorologiche 
e dalla luce solare. Il funzionamento di questo tipo di satellite si basa sull'uso di  onde radar 
che vengono inviate verso la Terra. Questi impulsi elettromagnetici rimbalzano sul terreno e sugli 
oggetti come edifici o vegetazione e tornano al satellite. Quest'ultimo analizzando il segnale di 
ritorno riesce ad ottenere informazioni sia sull'intensità del riflesso sia sul tempo impiegato 
dal segnale per tornare, dati fondamentali per ricostruire l'immagine del territorio. Il punto 
di forza del SAR è l'apertura sintetica. Poichè il satellite si muove lungo la sua orbita, i 
segnali raccolti in posizioni diverse vengono combinati insieme. Questo processo permette di 
simulare un'antenna molto più grande di quella reale, ottenendo così immagini ad altissima 
risoluzione, molto piu dettagliate di quelle che un radar di dimensioni fisiche limitate potrebbe 
generare da solo. In pratica, il movimento del satellite trasforma un radar relativamente piccolo 
in uno strumento potentissimo per osservare il pianeta. L'immagine così generata però presenta un 
particolare tipo di rumore. Quest'ultimo si forma quando un impulso radar colpisce il terreno, 
questo non riflette semplicemente un segnale uniforme. In realtà, il segnale viene riflesso da 
moltissimi piccoli scatter presenti sulla superficie come foglie, rocce o edifici. Tutti questi 
ritorni interferiscono tra di loro, sommando le onde con fasi diverse. Il risultato di questa 
interferenza prende il nome di Speckle. Questo tipo di rumore non è un errore del satellite o 
del radar, ma una caratteristica intrinseca del tipo di misura e si presenta con un pattern granuloso
che rende l'immagine difficle da interpretare ed analizzare. Il processo di riduzione dello speckle 
prende il nome di despeckling. Quest'ultimo cerca di smussare o filtrare il rumore granulare senza 
però perdere le informazioni reali presenti nell'immagine. In letteratura vi sono molteplici 
approcci: alcui si basano su filtri spaziali che analizzano i pixel vicini, altri usano tecniche 
più sofisticate come statistica multivarianza o metodi di deep learning. Ogni approccio ha i suoi
punti di forza e le sue lacune sulla base del tipo di ambiente rappresentato nell'immagine. Lo scopo
di questa tesi è cercare di unire i punti di forza di alcuni modelli in modo da ottenere l'immagine 
con il despeckling più accurato possibile
\medskip

