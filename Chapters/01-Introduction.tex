% !TEX root = ../Thesis.tex

\chapter{Introduzione}
I satelliti SAR (Synthetic Aperture Radar) sono satelliti dotati di un radar ad apertura sintetica che permette
loro di acquisire immagini della superficie terrestre indipendentemente dalle 
condizioni meteorologiche e dalla luce solare. 
Missioni e piattaforme di riferimento includono Sentinel-1 (ESA) \cite{esa_sentinel1}, TerraSAR-X (DLR), 
COSMO-SkyMed (ASI) e RADARSAT.
I satelliti SAR, grazie a questa loro 
capacita, trovano applicazione in molteplici contesti. In ambito geologico \cite{nhess-20-2379-2020}, 
sono impiegati per il monitoraggio del suolo e dei processi 
geomorfologici, consentendo la mappatura di foreste, deserti e aree soggette a 
trasformazioni ambientali. Inoltre, risultano particolarmente efficaci nell’analisi 
dei fenomeni di deforestazione attraverso il rilevamento dei cambiamenti nella 
copertura boschiva. In ambito marittimo \cite{ESA_TrackingMaritimeTraffic}, permettono di localizzare navi anche in condizioni 
meteorologiche avverse e di rilevare sversamenti di petrolio o altre sostanze 
inquinanti. In ambito di infrastrutture e urbanistica \cite{esa_sentinel1}, vengono utilizzati per misurare gli 
spostamenti del terreno e delle aree urbane, oltre che per il controllo di dighe, 
ponti e ferrovie, e per l’osservazione dello sviluppo delle città. 
L'immagine generata dal satellite però presenta un particolare tipo di rumore. Quest'ultimo si forma quando un impulso radar colpisce il terreno, 
questo non riflette semplicemente un segnale uniforme. In realtà, il segnale viene riflesso da 
moltissimi piccoli scatter presenti sulla superficie come foglie, rocce o edifici. 
Tutti questi ritorni interferiscono tra di loro, sommando le onde con fasi diverse. Il risultato di questa 
interferenza prende il nome di Speckle. Questo tipo di rumore non è un errore del satellite o 
del radar, ma una caratteristica intrinseca del tipo di misura e si presenta con un pattern granuloso
che rende l'immagine difficile da interpretare ed analizzare.
Il processo di riduzione dello speckle 
prende il nome di despeckling \cite{1097762}. Quest'ultimo cerca di smussare o filtrare il rumore granulare senza 
però perdere le informazioni reali presenti nell'immagine. In letteratura vi sono molteplici 
approcci: alcui si basano su filtri spaziali che analizzano i pixel vicini, altri usano tecniche 
più sofisticate come metodi di deep learning. Ogni approccio ha i suoi
punti di forza e le sue lacune sulla base del tipo di ambiente rappresentato nell'immagine. 
Lo scopo di questa tesi è combinare i punti di forza di diversi modelli di despeckling, 
fondendo tra loro i risultati in modo da selezionare, per ciascuna regione dell’immagine, le parti 
migliori di ogni output. L’obiettivo finale è ottenere un’immagine con il despeckling più accurato possibile.
Per valutare questa ipotesi, sono stati analizzati e confrontati diversi approcci, al fine di individuare la strategia più efficace per la fusione delle immagini.
Un primo approccio per ottenere ciò consiste nell' utlizzare tecniche di machine learning
per predire la qualità di un immagine denoised attraverso una mappa di qualità. Ad ogni modello,
è associata una mappa che indica, pixel per pixel dove il modello ha funzionato meglio e 
dove invece peggio. La fusione avviene tramite la media pesata dove i relativi pesi 
sono le mappe di qualità. Questo approccio però non sfrutta al massimo i punti di forza 
di ogni singola immagine despeckled portando ad un risulato finale non soddisfacente, 
in quanto la qualità del denoising viene stimata concentrandosi sul singolo pixel senza 
guardare i vicini. Un secondo approccio più efficiente è basato sull’attenzione. Invece 
di utilizzare mappe di qualità che determinano la bontà del denoising di un singolo 
pixel, si utilizzano meccanismi basati sulla self e cross attention. Questo consente di andare oltre 
la valutazione locale pixel per pixel, mettendo in relazione l’informazione proveniente 
da più immagini despeckled e valorizzando i dettagli complementari.
\medskip

